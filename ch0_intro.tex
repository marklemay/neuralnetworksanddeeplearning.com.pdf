\documentclass[a4paper,twoside,10pt]{book}
\include{header}
\begin{document}
\chapter*{What this book is about}
\addcontentsline{toc}{chapter}{What this book is about}
\markboth{What this book is about}{What this book is about} % to avoid "0. Introduction" in the header

Until 2006 we didn't know how to train neural networks well. In 2006 techniques for learning deep neural networks were discoved.  Today deep neural networks and deep learning achieve outstanding performance on many important problems in computer vision, speech recognition, and natural language processing. They're being deployed on a large scale by companies such as Google, Microsoft, and Facebook.

The purpose of this book is to help you master the core concepts of neural networks, including modern techniques for deep learning. After working through the book you will have written code that uses neural networks and deep learning to solve complex pattern recognition problems. And you will have a foundation to use neural networks and deep learning to attack problems of your own devising.

\section*{A principle-oriented approach}
One conviction underlying the book is that it's better to obtain a solid understanding of the core principles of neural networks and deep learning, rather than a hazy understanding of a long laundry list of ideas. If you've understood the core ideas well, you can rapidly understand other new material.

\section*{A hands-on approach}
We'll learn the core principles behind neural networks and deep learning by attacking a concrete problem: the problem of teaching a computer to recognize handwritten digits. This problem is extremely difficult to solve using the conventional approach to programming. And yet, as we'll see, it can be solved pretty well using a simple neural network, with just a few tens of lines of code, and no special libraries. What's more, we'll improve the program through many iterations, gradually incorporating more and more of the core ideas about neural networks and deep learning.

It's rare for a book to aim to be both principle-oriented and hands-on. But I believe you'll learn best if we build out the fundamental ideas of neural networks. We'll develop living code, not just abstract theory, code which you can explore and extend. This way you'll understand the fundamentals, both in theory and practice, and be well set to add further to your knowledge.
\end{document}
