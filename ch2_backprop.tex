\documentclass[a4paper,twoside,10pt]{book}
\include{header}
\begin{document}
\setcounter{chapter}{1}
\chapter{How the backpropagation algorithm works}
In the last chapter we saw how neural networks can learn their weights and biases using the gradient descent algorithm.  This chapter will explain a fast algorithm for computing such gradients: \textit{backpropagation}.

At the heart of backpropagation is an expression for the partial derivative $\partial{}C/\partial{}w$ of the cost function $C$ with respect to any weight $w$ (or bias $b$) in the network. The expression tells us how quickly the cost changes when we change the weights and biases.

\section{A matrix-based approach to computing the output from a neural network}
Let's review a fast matrix-based algorithm to compute the output from a neural network (section \ref{sec:1.7}).

Fisrt, a notation for weights in the network. $w^l_{jk}$ will denote the weight for the connection from the $k$-th neuron in the $(l-1)$-th layer to the $j$-th neuron in the $l$-th layer. The diagram below shows the weight on a connection from the fourth neuron in the second layer to the second neuron in the third layer of a network:

\begin{center}
	\includegraphics[width=0.7\linewidth]{./figures/ch2/tikz16.png}
\end{center}

%% break into mutiple paragraphs
We use a similar notation for the network's biases and activations. Explicitly, we use $b^l_j$ for the bias of the $j$-th neuron in the $l$-th layer. And we use $a^l_j$ for the activation of the $j$-th neuron in the $l$-th layer. The following diagram shows examples of these notations in use:
\begin{center}
\includegraphics[width=0.45\linewidth]{./figures/ch2/tikz17.png}
\end{center}
With these notations, the activation $a^l_j$ of the $j$-th neuron in the $l$-th layer is related to the activations in the $(l-1)$-th layer by the equation (compare Equation~(\ref{eq:4}))

\begin{equation}
	a^{l}_j = \sigma\left( \sum_k w^{l}_{jk} a^{l-1}_k + b^l_j \right),\tag{23}\label{eq:23}
\end{equation}
where the sum is over all neurons $k$ in the $(l-1)$-th layer. To rewrite this expression in a matrix form, define a \textit{weight matrix} $w^l$ for each layer, $l$. The entries of $w^l$ are the weights connecting to the $l$-th layer of neurons. That is, the entry in the $j$-th row and $k$-th column is $w^l_{jk}$. Similarly, for each layer $l$ we define a \textit{bias vector}, $b^l$. The components of the bias vector are just the values $b^l_j$, one component for each neuron in the $l$-th layer. And an activation vector $a^l$ whose components are the activations $a^l_j$.
The last ingredient we need to rewrite (\ref{eq:23}) in a matrix form is a way to ``vectorize'' a function, such as $\sigma$. We want to apply a function such as $\sigma$ to every element in a vector $v$. We use the notation $\sigma(v)$ to denote elementwise application of a function. That is, the components of $\sigma(v)$ are $\sigma(v)_j = \sigma(v_j)$. As an example, if we have the function $f(x)=x^2$ then the vectorized form of $f$ has the effect
\begin{equation}
f\left(\left[ \begin{array}{c} 2 \\ 3 \end{array} \right] \right)
= \left[ \begin{array}{c} f(2) \\ f(3) \end{array} \right]
= \left[ \begin{array}{c} 4 \\ 9 \end{array} \right],
\tag{24}\label{eq:24}
\end{equation}
that is, the vectorized $f$ just squares every element of the vector.

With these notations in mind, Equation~(\ref{eq:23}) can be rewritten in the beautiful and compact vectorized form
\begin{equation}
a^{l} = \sigma(w^l a^{l-1}+b^l).
\tag{25}\label{eq:25}
\end{equation}
Just apply the weight matrix to the activations, then add the bias vector, and finally apply the $\sigma$ function. This expression is useful because most matrix libraries provide fast ways of implementing matrix multiplication, vector addition, and vectorization (see \ref{sec:1.7}).

When using Equation~(\ref{eq:25}) to compute $a^l$, we compute the intermediate quantity $z^l = w^l a^{l-1}+b^l$. We call $z^l$ the weighted input to the neurons in layer $l$. Equation~(\ref{eq:25}) is sometimes written in terms of the weighted input, as $a^l=\sigma(z^l)$.

\section{The two assumptions we need about the cost function}
The goal of backpropagation is to compute the partial derivatives $\partial{}C/\partial{}w$ and $\partial{}C/\partial{}b$ of the cost function $C$ with respect to any weight $w$ or bias $b$ in the network. For backpropagation to work we need to make two assumptions about the cost function:
\begin{enumerate}
  \item The cost function can be written as an average $C=\frac1n\sum_xC_x$ over cost functions $C_x$ for individual training examples, $x$. %% linear combo
  \item The cost function can be written as a function of the outputs from the neural network
\end{enumerate}


Consider the quadratic cost function from last chapter (c.f. Equation~(\ref{eq:6})),
\begin{equation}
C = \frac{1}{2n} \sum_x \left\|y(x)-a^L(x)\right\|^2,
\tag{26}\label{eq:26}
\end{equation}
where: $n$ is the number of training examples; the sum is over individual training examples, $x$; $y(x)$ is the corresponding desired output; $L$ denotes the number of layers in the network; and $a^L(x)$ is the vector of activations output from the network when $x$ is input.

The first assumption we need for backpropagation is that the cost function can be written as an average. This is the case for the quadratic cost function, where the cost for a single training example is $C_x=\frac12\|y-a^L\|^2$. This assumption will also hold true for all the other cost functions in this book.

We need this assumption because what backpropagation lets us do is compute the partial derivatives $\partial{}C_x/\partial{}w$ and $\partial{}C_x/\partial{}b$ for a single training example. We then recover $\partial{}C/\partial{}w$ and $\partial{}C/\partial{}b$ by averaging over training examples. In fact, with this assumption in mind, we'll suppose the training example $x$ has been fixed, and drop the $x$ subscript, writing the cost $C_x$ as $C$. %% not ideal notation

The second assumption about the cost function is that it can be written as a function of the outputs from the neural network:

\begin{center}
\includegraphics[width=0.7\linewidth]{./figures/ch2/tikz18.png}
\end{center}
The quadratic cost function satisfies this requirement, since the quadratic cost for a single training example $x$ may be written as
\begin{equation}
C = \frac{1}{2} \left\|y-a^L\right\|^2 = \frac{1}{2} \sum_j \left(y_j-a^L_j\right)^2,
\tag{27}\label{eq:27}
\end{equation}
and thus is a function of the output activations.

\section{The Hadamard product, $s\odot{}t$}
Suppose $s$ and $t$ are two vectors of the same dimension. Then we use $s\odot{}t$ to denote the elementwise product of the two vectors. Thus the components of $s\odot{}t$ are $(s \odot t)_j = s_j t_j$. As an example,
\begin{equation}
	\left[\begin{array}{c} 1 \\ 2 \end{array}\right] 
	\odot \left[\begin{array}{c} 3 \\ 4\end{array} \right]
	= \left[ \begin{array}{c} 1 * 3 \\ 2 * 4 \end{array} \right]
	= \left[ \begin{array}{c} 3 \\ 8 \end{array} \right].
	\tag{28}\label{eq:28}
\end{equation}
This kind of elementwise multiplication is called the \textit{Hadamard product} or \textit{Schur product}. Good matrix libraries usually provide fast implementations of the Hadamard product, and that comes in handy when implementing backpropagation.

\section{The four fundamental equations behind backpropagation}
\label{sec:2.4}
Backpropagation is about understanding how changing the weights and biases in a network changes the cost function. Ultimately, this means computing the partial derivatives $\partial{}C/\partial{}w^l_{jk}$ and $\partial{}C/\partial{}b^l_j$. To compute those, we introduce an intermediate quantity, $\delta^l_j$, which we call the \textit{error} in the $j$-th neuron in the $l$-th layer. Backpropagation will give us a procedure to compute the error $\delta^l_j$, and then will relate $\delta^l_j$ to $\partial{}C/\partial{}w^l_{jk}$ and $\partial{}C/\partial{}b^l_j$.

To understand how the error is defined, imagine there is a demon in our neural network:

\begin{center}
	\includegraphics[width=0.6\linewidth]{./figures/ch2/tikz19.png}
\end{center}
The demon sits at the $j$-th neuron in layer $l$. As the input to the neuron comes in, the demon messes with the neuron's operation. It adds a little change $\Delta{}z^l_j$ to the neuron's weighted input, so that instead of outputting $\sigma(z^l_j)$, the neuron instead outputs $\sigma(z^l_j+\Delta{}z^l_j)$. This change propagates through later layers in the network, finally causing the overall cost to change by an amount $\frac{\partial{}C}{\partial{}z^l_j}\Delta{}z^l_j$.

Now, this demon is a good demon, and is trying to help you improve the cost, i.e., they're trying to find a $\Delta{}z^l_j$ which makes the cost smaller. Suppose $\partial{}C/\partial{}z^l_j$ has a large value (either positive or negative). Then the demon can lower the cost quite a bit by choosing $\Delta{}z^l_j$ to have the opposite sign to $\partial{}C/\partial{}z^l_j$. By contrast, if $\partial{}C/\partial{}z^l_j$ is close to zero, then the demon can't improve the cost much at all by perturbing the weighted input $z^l_j$. So far as the demon can tell, the neuron is already pretty near optimal. And so there's a heuristic sense in which $\partial{}C/\partial{}z^l_j$ is a measure of the error in the neuron.

Motivated by this story, we define the error $\delta^l_j$ of neuron $j$ in layer $l$ by
\begin{equation}
	\delta^l_j = \frac{\partial{}C}{\partial{}z^l_j}.\tag{29}\label{eq:29}
\end{equation}
We use $\delta{}^l$ to denote the vector of errors associated with layer $l$. Backpropagation will give us a way of computing $\delta{}^l$ for every layer, and then relating those errors to the quantities of interest, $\partial{}C/\partial{}w^l_{jk}$ and $\partial{}C/\partial{}b^l_j$.

\textbf{Plan of attack}: Backpropagation is based around four fundamental equations. Together, those equations give us a way of computing both the error $\delta{}^l$ and the gradient of the cost function.

Here's a preview of the ways we'll delve more deeply into the equations in this chapter. 
\begin{enumerate}
  \item I'll give a \hyperref[sec:2.5]{short proof of the equations}, which helps explain why they are true;
  \item we'll restate the equations in algorithmic form as \hyperref[sec:2.6]{pseudocode}, and see how the pseudocode can be implemented as real, \hyperref[sec:2.7]{running Python code};
  \item and, in the final section of the chapter, we'll develop an intuitive picture of what the backpropagation equations mean, and how someone \hyperref[sec:2.9]{might discover them from scratch}.
\end{enumerate}
Along the way we'll return repeatedly to the four fundamental equations.

\textbf{An equation for the error in the output layer,} $\delta^L$: The components of $\delta^L$ are given by
\begin{equation}
	\delta^L_j = \frac{\partial C}{\partial a^L_j} \sigma'(z^L_j).
	\tag{BP1}\label{eq:BP1}
\end{equation}
The term, $\partial{}C/\partial{}a^L_j$, measures how fast the cost is changing as a function of the $j$-th output activation. If, for example, $C$ doesn't depend much on a particular output neuron, $j$, then $\delta^L_j$ will be small, which is what we'd expect. The term, $\sigma'(z^L_j)$, measures how fast the activation function $\sigma$ is changing at $z^L_j$.

Everything in Eq.~(\ref{eq:BP1}) is easily computed. In particular, we compute $z^L_j$ while computing the behaviour of the network, and it's only a small additional overhead to compute $\sigma'(z^L_j)$. Provided the cost function is known there should be little trouble computing $\partial{}C/\partial{}a^L_j$. For example, if we're using the quadratic cost function $C=\frac12\sum_j(y_j-a^L_j)^2$, so $\partial{}C/\partial{}a^L_j=(a^L_j-y_j)$.

Equation~(\ref{eq:BP1}) is a componentwise expression for $\delta^L$. The matrix-based form is
%$\delta^l$=∇aC⊙$\sigma$′(zL).(BP1a)
\begin{equation}
\delta^L = \nabla_a C \odot \sigma'(z^L).
\tag{BP1a}\label{eq:BP1a}
\end{equation}
Here, $\nabla_a C$ is the vector whose components are the partial derivatives $\partial{}C/\partial{}a^L_j$. You can think of $\nabla_a C$ as expressing the rate of change of $C$ with respect to the output activations. As an example, in the case of the quadratic cost we have $\nabla_a C =(a^L-y)$, and so the matrix-based form of (\ref{eq:BP1}) becomes
\begin{equation}
\delta^L = (a^L-y) \odot \sigma'(z^L).
\tag{30}\label{eq:30}
\end{equation}

\textbf{An equation for the error $\delta^l$ in terms of the error in the next layer}, $\delta^{l+1}$:
\begin{equation}
	\delta^l = \left((w^{l+1})^T \delta^{l+1}\right) \odot \sigma'(z^l),
	\tag{BP2}\label{eq:BP2}
\end{equation}
%$\delta^l$=(($w^{l+1}$)T$\delta^{l+1}$)⊙$\sigma$′(zl),(\ref{eq:BP2})
where $(w^{l+1})^T$ is the transpose of the weight matrix $w^{l+1}$ for the ($l$+1)-th layer. Suppose we know the error $\delta^{l+1}$ at the ($l$+1)-th layer. When we apply the transpose weight matrix, $(w^{l+1})^T$, we can think of this as moving the error \textit{backward} through the network, giving us some sort of the error at the output of the $l$-th layer. We then take the Hadamard product $\odot \sigma'(z^l)$. This moves the error backward through the activation function in layer $l$, giving us the error $\delta^l$ in the weighted input to layer $l$.

By combining (\ref{eq:BP2}) with (\ref{eq:BP1}) we can compute the error $\delta^l$ for any layer in the network. We start by using (\ref{eq:BP1}) to compute $\delta^L$, then apply Equation~(\ref{eq:BP2}) to compute $\delta^{L-1}$, then Equation~(\ref{eq:BP2}) again to compute $\delta^{L-2}$, and so on, all the way back through the network.

\textbf{An equation for the rate of change of the cost with respect to any bias in the network:}
%\partial{}C\partial{}b^l_j=\delta^l_j.(\ref{eq:BP3})
\begin{equation}
\frac{\partial C}{\partial b^l_j} = \delta^l_j.
\tag{BP3}\label{eq:BP3}
\end{equation}
That is, the error $\delta^l_j$ is \textit{exactly equal} to the rate of change $\partial{}C/\partial{}b^l_j$. This is great news, since (\ref{eq:BP1}) and (\ref{eq:BP2}) have already told us how to compute $\delta^l_j$. We can rewrite (\ref{eq:BP3}) as
\begin{equation}
\frac{\partial C}{\partial b} = \delta,\tag{31}\label{eq:31} % TODO use consitent formula names: BP3b?, or even better BP3i for the indexed version, BP3 for non indexed?
\end{equation}
where it is understood that $\delta$ is being evaluated at the same neuron as the bias $b$.  % TODO never mind just put the indexes back in and ignore this sentence

\textbf{An equation for the rate of change of the cost with respect to any weight in the network:}
%\partial{}C\partial{}w^l_{jk}=al-1k\delta^l_j.(\ref{eq:BP4})
\begin{equation}
\frac{\partial C}{\partial w^l_{jk}} = a^{l-1}_k \delta^l_j.
\tag{BP4}\label{eq:BP4}
\end{equation}
This tells us how to compute the partial derivatives $\partial{}C/\partial{}w^l_{jk}$ in terms of the quantities $\delta^l$ and $a^{l-1}$, which we already know how to compute. The equation can be rewritten in a less index-heavy notation as
%\partial{}C\partial{}w=ain\Delta{}out,(32)
\begin{equation}
\frac{\partial C}{\partial w} = a_{\rm in} \delta_{\rm out},
\tag{32}\label{eq:32}
\end{equation}
where it's understood that $a_\mathrm{in}$ is the activation of the neuron input to the weight $w$, and $\delta_\mathrm{out}$ is the error of the neuron output from the weight $w$. Zooming in to look at just the weight $w$, and the two neurons connected by that weight, we can depict this as:

\begin{center}
	\includegraphics[width=0.35\linewidth]{./figures/ch2/tikz20.png}
\end{center}
A nice consequence of Equation~(\ref{eq:32}) is that when the activation $a_\mathrm{in}$ is small, $a_\mathrm{in}\approx0$, the gradient term $\partial{}C/\partial{}w$ will also tend to be small. In this case, we'll say the weight \textit{learns slowly}, meaning that it's not changing much during gradient descent. Weights output from low-activation neurons learn slowly.

There are other insights along these lines which can be obtained from (\ref{eq:BP1})--(\ref{eq:BP4}). Let's start by looking at the output layer. Consider the term $\sigma'(z^L_j)$ in (\ref{eq:BP1}). Recall that the $\sigma$ function becomes very flat when $\sigma(z^L_j)$ is approximately $0$ or $1$. When this occurs we will have $\sigma'(z^L_j)\approx0$. So the weight in the final layer will learn slowly if the output neuron is either low activation ($\approx0$) or high activation ($\approx1$). In this case neuron is \textit{saturated} and, as a result, the weight has stopped learning (or is learning slowly). Similar remarks hold also for the biases of output neuron. % TODO be specifc

We can obtain similar insights for earlier layers. Note the $\sigma'(z^l)$ term in (\ref{eq:BP2}). This means that $\delta^l_j$ is likely to get small if the neuron is near saturation. And this, in turn, means that any weights input to a saturated neuron will tend to learn slowly.

A weight will learn slowly if either the input neuron is low-activation, or if the output neuron has saturated, i.e., is either high- or low-activation.

The four fundamental equations turn out to hold for any activation function, not just the sigmoid function. So we can use these equations to design activation functions which have particular desired learning properties. As an example, suppose we were to choose a (non-sigmoid) activation function $\sigma$ so that $\sigma'$ is always positive, and never gets close to zero. That would prevent the slow-down of learning that occurs when ordinary sigmoid neurons saturate. Later we'll see examples where this kind of modification is made to the activation function. Keeping the four equations (\ref{eq:BP1})--(\ref{eq:BP4}) in mind can help explain why such modifications are tried, and what impact they can have.
\begin{center}
	\includegraphics[width=0.65\linewidth]{figures/ch2/tikz21.png}
\end{center}

\begin{exercize}{Problem}
	\item \textbf{Alternate presentation of the equations of backpropagation:} I've stated the equations of backpropagation (notably (\ref{eq:BP1}) and (\ref{eq:BP2})) using the Hadamard product. This presentation may be disconcerting if you're unused to the Hadamard product. There's an alternative approach, based on conventional matrix multiplication, which some readers may find enlightening.
	\begin{enumerate}
		\item[(1)]  Show that (\ref{eq:BP1}) may be rewritten as
		\begin{equation}
			\delta^L = \Sigma'(z^L) \nabla_a C,
			\tag{33}\label{eq:33}
		\end{equation}
		where $\Sigma'(z^L)$ is a square matrix whose diagonal entries are the values $\sigma'(z^L_j)$, and whose off-diagonal entries are zero. Note that this matrix acts on $\nabla_a C$ by conventional matrix multiplication.
		\item[(2)] Show that (\ref{eq:BP2}) may be rewritten as
		\begin{equation}
			\delta^l = \Sigma'(z^l) (w^{l+1})^T \delta^{l+1}.
			\tag{34}\label{eq:34}
		\end{equation}
		\item[(3)] By combining observations (1) and (2) show that
		\begin{equation}
			\delta^l = \Sigma'(z^l) (w^{l+1})^T \ldots \Sigma'(z^{L-1}) (w^L)^T \Sigma'(z^L) \nabla_a C
			\tag{35}\label{eq:35}
		\end{equation}
		For readers comfortable with matrix multiplication this equation may be easier to understand than (\ref{eq:BP1}) and (\ref{eq:BP2}).
	\end{enumerate}
\end{exercize}
\section{Proof of the four fundamental equations}
\label{sec:2.5}
We'll now prove the four fundamental equations (\ref{eq:BP1})--(\ref{eq:BP4}). All four are consequences of the chain rule from multivariable calculus. If you're comfortable with the chain rule, then I strongly encourage you to attempt the derivation yourself before reading on.

Let's begin with Equation~(\ref{eq:BP1}), which gives an expression for the output error, $\delta^l$. To prove this equation, recall that by definition
%$\delta^l$j=\partial{}C\partial{}z^L_j.(36)
\begin{equation}
	\delta^L_j = \frac{\partial C}{\partial z^L_j}.
	\tag{36}\label{eq:36}
\end{equation}
Applying the chain rule, we can re-express the partial derivative above in terms of partial derivatives with respect to the output activations,
\begin{equation}
	\delta^L_j = \sum_k \frac{\partial C}{\partial a^L_k} \frac{\partial a^L_k}{\partial z^L_j},
	\tag{37}\label{eq:37}
\end{equation}%$\delta^l$j=∑k\partial{}C\partial{}aLk\partial{}aLk\partial{}z^L_j,(37)
where the sum is over all neurons $k$ in the output layer. Of course, the output activation $a^L_k$ of the $k$-th neuron depends only on the weighted input $z^L_j$ for the $j$-th neuron when $k=j$. And so $\partial{}a^L_k/\partial{}z^L_j$ vanishes when $k\ne{}j$. As a result we can simplify the previous equation to
\begin{equation}
	\delta^L_j = \frac{\partial C}{\partial a^L_j} \frac{\partial a^L_j}{\partial z^L_j}.
	\tag{38}\label{eq:38}
\end{equation}%$\delta^l$j=\partial{}C\partial{}a^L_j\partial{}a^L_j\partial{}z^L_j.(38)
Recalling that $a^L_j=\sigma(z^L_j)$ the second term on the right can be written as $\sigma'(z^L_j)$, and the equation becomes
%$\delta^l$j=\partial{}C\partial{}a^L_j$\sigma$′(z^L_j),(39)
\begin{equation}
	\delta^L_j = \frac{\partial C}{\partial a^L_j} \sigma'(z^L_j),
	\tag{39}\label{eq:39}
\end{equation}
which is just (\ref{eq:BP1}), in component form.
%
Next, we'll prove (\ref{eq:BP2}), which gives an equation for the error $\delta^l$ in terms of the error in the next layer, $\delta^{l+1}$. To do this, we want to rewrite $\delta^l_j=\partial{}C/\partial{}z^l_j$ in terms of $\delta^{l+1}_k=\partial{}C/\partial{}z^{l+1}_k$. We can do this using the chain rule,
\begin{equation}
\delta^l_j =  \frac{\partial C}{\partial z^l_j} =  \sum_k \frac{\partial C}{\partial z^{l+1}_k} \frac{\partial z^{l+1}_k}{\partial z^l_j} = \sum_k \frac{\partial z^{l+1}_k}{\partial z^l_j} \delta^{l+1}_k,
\tag{42}\label{eq:42}
\end{equation}
%\delta^l_j===\partial{}C\partial{}z^l_j∑k\partial{}C\partial{}zl+1k\partial{}zl+1k\partial{}z^l_j∑k\partial{}zl+1k\partial{}z^l_j$\delta^{l+1}$k,(40)(41)(42)
where in the last line we have interchanged the two terms on the right-hand side, and substituted the definition of $\delta^{l+1}_k$. To evaluate the first term on the last line, note that
%zl+1k=∑j$w^{l+1}$kjalj+bl+1k=∑j$w^{l+1}$kj$\sigma$(z^l_j)+bl+1k.(43)
\begin{equation}
	z^{l+1}_k = \sum_j w^{l+1}_{kj} a^l_j +b^{l+1}_k = \sum_j w^{l+1}_{kj} \sigma(z^l_j) +b^{l+1}_k.
	\tag{43}\label{eq:43}
\end{equation}
Differentiating, we obtain
%\partial{}zl+1k\partial{}z^l_j=$w^{l+1}$kj$\sigma$′(z^l_j).(44)
\begin{equation}
	\frac{\partial z^{l+1}_k}{\partial z^l_j} = w^{l+1}_{kj} \sigma'(z^l_j).
	\tag{44}\label{eq:44}
\end{equation}
Substituting back into (\ref{eq:42}) we obtain
\begin{equation}
	\delta^l_j = \sum_k w^{l+1}_{kj}  \delta^{l+1}_k \sigma'(z^l_j).
	\tag{45}\label{eq:45}
\end{equation}%\delta^l_j=∑k$w^{l+1}$kj$\delta^{l+1}$k$\sigma$′(z^l_j).(45)
This is just (\ref{eq:BP2}) written in component form.

The final two equations we want to prove are (\ref{eq:BP3}) and (\ref{eq:BP4}). These also follow from the chain rule, in a manner similar to the proofs of the two equations above. I leave them to you as an exercise.

\begin{exercize}{Exercise}
	\item Prove Equations (\ref{eq:BP3}) and (\ref{eq:BP4}).
\end{exercize}
That completes the proof of the four fundamental equations of backpropagation. The proof may seem complicated. But it's really just the outcome of carefully applying the chain rule. A little less succinctly, we can think of backpropagation as a way of computing the gradient of the cost function by systematically applying the chain rule from multi-variable calculus. That's all there really is to backpropagation -- the rest is details.

\section{The backpropagation algorithm}
\label{sec:2.6}
The backpropagation equations provide us with a way of computing the gradient of the cost function. Let's explicitly write this out in the form of an algorithm:
\begin{enumerate}
\item \textbf{Input} $x$: Set the corresponding activation $a^1$ for the input layer.
\item \textbf{Feedforward:} For each $l=2,3,\ldots{},L$ compute $z^l=w^la^{l-1}+b^l$ and $a^l=\sigma(z_l)$.
\item \textbf{Output error} $\delta^L$: Compute the vector $\delta^{L} = \nabla_a C \odot \sigma'(z^L)$.
\item \textbf{Backpropagate the error}: For each $l = L-1, L-2,\ldots, 2$ compute $\delta^{l} = ((w^{l+1})^T \delta^{l+1}) \odot \sigma'(z^{l})$.
\item \textbf{Output}: The gradient of the cost function is given by $\frac{\partial{}C}{\partial{}w^l_{jk}}=a^{l-1}_k\delta^l_j$ and $\frac{\partial{}C}{\partial{}b^l_j}=\delta^l_j$.
\end{enumerate}
Examining the algorithm you can see why it's called\textit{ back}propagation. We compute the error vectors $\delta^l$ backward, starting from the final layer. It may seem peculiar that we're going through the network backward. But if you think about the proof of backpropagation, the backward movement is a consequence of the fact that the cost is a function of outputs from the network. To understand how the cost varies with earlier weights and biases we need to repeatedly apply the chain rule, working backward through the layers to obtain usable expressions.

\begin{exercize}{Exercises}
	\item \textbf{Backpropagation with a single modified neuron} Suppose we modify a single neuron in a feedforward network so that the output from the neuron is given by $f(\sum_j w_j x_j + b)$, where $f$ is some function other than the sigmoid. How should we modify the backpropagation algorithm in this case?
	\item \textbf{Backpropagation with linear neurons} Suppose we replace the usual non-linear $\sigma$ function with $\sigma(z)=z$ throughout the network. Rewrite the backpropagation algorithm for this case.
\end{exercize}
As I've described it above, the backpropagation algorithm computes the gradient of the cost function for a single training example, $C=C_x$. In practice, it's common to combine backpropagation with a learning algorithm such as stochastic gradient descent, in which we compute the gradient for many training examples. In particular, given a mini-batch of $m$ training examples, the following algorithm applies a gradient descent learning step based on that mini-batch:
\begin{enumerate}
\item Input a set of training examples
\item For each training example $x$: Set the corresponding input activation $a^{x,1}$, and perform the following steps:
	\begin{itemize} 
		\item Feedforward: For each l$=2,3,\ldots,L$ compute $z^{x,l}=w^la^{x,l-1}+b^l$ and $a^{x,l}=\sigma(z^{x,l})$.
		\item Output error $\delta^{x,L}$: Compute the vector $\delta^{x,L} = \nabla_a C_x \odot \sigma'(z^{x,L})$.
		\item Backpropagate the error: For each $l=L-1,L-2,\ldots,2$ compute $\delta^{x,l} = ((w^{l+1})^T \delta^{x,l+1}) \odot \sigma'(z^{x,l})$.
	\end{itemize}
\item Gradient descent: For each $l=L,L-1,\ldots,2$ update the weights according to the rule $w^l \to
w^l-\frac{\eta}{m} \sum_x \delta^{x,l} (a^{x,l-1})^T$, and the biases according to the rule $b^l \to b^l-\frac{\eta}{m}\sum_x \delta^{x,l}$.
\end{enumerate}
Of course, to implement stochastic gradient descent in practice you also need an outer loop generating mini-batches of training examples, and an outer loop stepping through multiple epochs of training.
\section{The code for backpropagation}
\label{sec:2.7}
Having understood backpropagation in the abstract, we can now understand the code used in the last chapter to implement backpropagation. Recall from that chapter that the code was contained in the \inline{update_mini_batch} and \inline{backprop} methods of the \inline{Network} class. The code for these methods is a direct translation of the algorithm described above. In particular, the \inline{update_mini_batch}  method updates the \inline{Network}'s weights and biases by computing the gradient for the current \inline{mini_batch} of training examples:
\begin{lstlisting}
class Network(object):
...
	def update_mini_batch(self, mini_batch, eta):
		"""Update the network's weights and biases by applying
		gradient descent using backpropagation to a single mini batch.
		The "mini_batch" is a list of tuples "(x, y)", and "eta"
		is the learning rate."""
		nabla_b = [np.zeros(b.shape) for b in self.biases]
		nabla_w = [np.zeros(w.shape) for w in self.weights]
		for x, y in mini_batch:
			delta_nabla_b, delta_nabla_w = self.backprop(x, y)
			nabla_b = [nb+dnb for nb, dnb in zip(nabla_b, delta_nabla_b)]
			nabla_w = [nw+dnw for nw, dnw in zip(nabla_w, delta_nabla_w)]
		self.weights = [w-(eta/len(mini_batch))*nw 
						for w, nw in zip(self.weights, nabla_w)]
		self.biases = [b-(eta/len(mini_batch))*nb 
						for b, nb in zip(self.biases, nabla_b)]

\end{lstlisting}
Most of the work is done by the line \inline{delta_nabla_b}, \inline{delta_nabla_w = self.backprop(x, y)} which uses the \inline{backprop} method to figure out the partial derivatives $\partial{}C_x/\partial{}b^l_j$ and $\partial{}C_x/\partial{}w^l_{jk}$. The \inline{backprop} method follows the algorithm in the last section closely. There is one small change -- we use a slightly different approach to indexing the layers. This change is made to take advantage of a feature of Python, namely the use of negative list indices to count backward from the end of a list, so, e.g., \inline{l[-3]} is the third last entry in a list \inline{l}. The code for \inline{backprop} is below, together with a few helper functions, which are used to compute the $\sigma$ function, the derivative $\sigma'$, and the derivative of the cost function. With these inclusions you should be able to understand the code in a self-contained way.

\begin{lstlisting}
class Network(object):
...
	def backprop(self, x, y):
		"""Return a tuple "(nabla_b, nabla_w)" representing the
		gradient for the cost function C_x.  "nabla_b" and
		"nabla_w" are layer-by-layer lists of numpy arrays, similar
		to "self.biases" and "self.weights"."""
		nabla_b = [np.zeros(b.shape) for b in self.biases]
		nabla_w = [np.zeros(w.shape) for w in self.weights]
		# feedforward
		activation = x
		activations = [x] # list to store all the activations, layer by layer
		zs = [] # list to store all the z vectors, layer by layer
		for b, w in zip(self.biases, self.weights):
			z = np.dot(w, activation)+b
			zs.append(z)
			activation = sigmoid(z)
			activations.append(activation)
		# backward pass
		delta = self.cost_derivative(activations[-1], y) * sigmoid_prime(zs[-1])
		nabla_b[-1] = delta
		nabla_w[-1] = np.dot(delta, activations[-2].transpose())
		# Note that the variable l in the loop below is used a little
		# differently to the notation in Chapter 2 of the book.  Here,
		# l = 1 means the last layer of neurons, l = 2 is the
		# second-last layer, and so on.  It's a renumbering of the
		# scheme in the book, used here to take advantage of the fact
		# that Python can use negative indices in lists.
		for l in xrange(2, self.num_layers):
			z = zs[-l]
			sp = sigmoid_prime(z)
			delta = np.dot(self.weights[-l+1].transpose(), delta) * sp
			nabla_b[-l] = delta
			nabla_w[-l] = np.dot(delta, activations[-l-1].transpose())
		return (nabla_b, nabla_w)
...
	def cost_derivative(self, output_activations, y):
		"""Return the vector of partial derivatives \partial{} C_x /
		\partial{} a for the output activations."""
		return (output_activations-y) 
	def sigmoid(z):
		"""The sigmoid function."""
		return 1.0/(1.0+np.exp(-z))
	def sigmoid_prime(z):
		"""Derivative of the sigmoid function."""
		return sigmoid(z)*(1-sigmoid(z))
\end{lstlisting}


\begin{exercize}{Problem}
\label{backprop_over_minibatch}
\item \textbf{Fully matrix-based approach to backpropagation over a mini-batch} Our implementation of stochastic gradient descent loops over training examples in a mini-batch. It's possible to modify the backpropagation algorithm so that it computes the gradients for all training examples in a mini-batch simultaneously. The idea is that instead of beginning with a single input vector, $x$, we can begin with a matrix $X=[x_1x_2\ldots{}x_m]$ whose columns are the vectors in the mini-batch. We forward-propagate by multiplying by the weight matrices, adding a suitable matrix for the bias terms, and applying the sigmoid function everywhere. We backpropagate along similar lines. Explicitly write out pseudocode for this approach to the backpropagation algorithm. Modify \inline{network.py} so that it uses this fully matrix-based approach. The advantage of this approach is that it takes full advantage of modern libraries for linear algebra. As a result it can be quite a bit faster than looping over the mini-batch. (On my laptop, for example, the speedup is about a factor of two when run on MNIST classification problems like those we considered in the last chapter.) In practice, all serious libraries for backpropagation use this fully matrix-based approach or some variant.
\end{exercize}
\section{In what sense is backpropagation a fast algorithm?}
\label{sec:2.8}
Let's consider another approach to computing the gradient. Imagine it's the early days of neural networks research. Maybe it's the 1950s or 1960s, and you're the first person in the world to think of using gradient descent to learn! But to make the idea work you need a way of computing the gradient of the cost function. You decide to regard the cost as a function of the weights $C=C(w)$ alone (we'll get back to the biases in a moment). You number the weights $w_1,w_2,\ldots$, and want to compute $\partial{}C/\partial{}w_j$ for some particular weight $w_j$. An obvious way of doing that is to use the approximation
\begin{equation}
	\frac{\partial C}{\partial w_{j}} \approx \frac{C(w+\epsilon e_j)-C(w)}{\epsilon},
	\tag{46}\label{eq:46}
\end{equation}
%\partial{}C\partial{}wj≈C(w+\epsilon{}ej)-C(w)\epsilon{},(46)
where $\epsilon>0$ is a small positive number, and $e_j$ is the unit vector in the $j$-th direction. In other words, we can estimate $\partial{}C/\partial{}w_j$ by computing the cost $C$ for two slightly different values of $w_j$, and then applying Equation~(\ref{eq:46}). The same idea will let us compute the partial derivatives $\partial{}C/\partial{}b$ with respect to the biases.

This approach looks very promising. It's simple conceptually, and extremely easy to implement, using just a few lines of code. Certainly, it looks much more promising than the idea of using the chain rule to compute the gradient!

Unfortunately, while this approach appears promising, when you implement the code it turns out to be extremely slow. To understand why, imagine we have a million weights in our network. Then for each distinct weight $w_j$ we need to compute $C(w+\epsilon{}e_j)$ in order to compute $\partial{}C/\partial{}w_j$. That means that to compute the gradient we need to compute the cost function a million different times, requiring a million forward passes through the network (per training example). We need to compute $C(w)$ as well, so that's a total of a million and one passes through the network.

What's clever about backpropagation is that it enables us to simultaneously compute all the partial derivatives $\partial{}C/\partial{}w_j$ using just one forward pass through the network, followed by one backward pass through the network. Roughly speaking, the computational cost of the backward pass is about the same as the forward pass\footnote{This should be plausible, but it requires some analysis to make a careful statement. It's plausible because the dominant computational cost in the forward pass is multiplying by the weight matrices, while in the backward pass it's multiplying by the transposes of the weight matrices.}. And so the total cost of backpropagation is roughly the same as making just two forward passes through the network. Compare that to the million and one forward passes we needed for the approach based on (\ref{eq:46})! And so even though backpropagation appears superficially more complex than the approach based on (\ref{eq:46}), it's actually much, much faster.

This speedup was first fully appreciated in 1986, and it greatly expanded the range of problems that neural networks could solve. That, in turn, caused a rush of people using neural networks. Of course, backpropagation is not a panacea. Even in the late 1980s people ran up against limits, especially when attempting to use backpropagation to train deep neural networks, i.e., networks with many hidden layers. Later in the book we'll see how modern computers and some clever new ideas now make it possible to use backpropagation to train such deep neural networks.
%
\section{Backpropagation: the big picture}
\label{sec:2.9}
As I've explained it, backpropagation presents two mysteries. First, what's the algorithm really doing? We've developed a picture of the error being backpropagated from the output. But can we go any deeper, and build up more intuition about what is going on when we do all these matrix and vector multiplications? The second mystery is how someone could ever have discovered backpropagation in the first place? It's one thing to follow the steps in an algorithm, or even to follow the proof that the algorithm works. But that doesn't mean you understand the problem so well that you could have discovered the algorithm in the first place. Is there a plausible line of reasoning that could have led you to discover the backpropagation algorithm?

Let's imagine that we've made a small change $\Delta{}w^l_{jk}$ to some weight in the network, $w^l_{jk}$:
\begin{center}
	\includegraphics[scale=0.5]{./figures/ch2/tikz22.png}
\end{center}
That change in weight will cause a change in the output activation from the corresponding neuron:
\begin{center}
	\includegraphics[scale=0.5]{./figures/ch2/tikz23.png}
\end{center}
That, in turn, will cause a change in all the activations in the next layer:
\begin{center}
	\includegraphics[scale=0.5]{./figures/ch2/tikz24.png}
\end{center}
Those changes will in turn cause changes in the next layer, and then the next, and so on all the way through to causing a change in the final layer, and then in the cost function:
\begin{center}
	\includegraphics[scale=0.5]{./figures/ch2/tikz25.png}
\end{center}
The change $\Delta{}C$ in the cost is related to the change $\Delta{}w^l_{jk}$ in the weight by the equation
\begin{equation}
	\Delta C \approx \frac{\partial C}{\partial w^l_{jk}} \Delta w^l_{jk}.
	\tag{47}\label{eq:47}
\end{equation}
%\Delta{}C≈\partial{}C\partial{}w^l_{jk}\Delta{}w^l_{jk}.(47)
This suggests that a possible approach to computing $\partial{}C/\partial{}w^l_{jk}$ is to carefully track how a small change in $w^l_{jk}$ propagates to cause a small change in $C$. If we can do that, being careful to express everything along the way in terms of easily computable quantities, then we should be able to compute $\partial{}C/\partial{}w^l_{jk}$.

Let's try to carry this out. The change $\Delta{}w^l_{jk}$ causes a small change $\Delta{}a^l_j$ in the activation of the $j$-th neuron in the $l$-th layer. This change is given by
%\Delta{}alj≈\partial{}alj\partial{}w^l_{jk}\Delta{}w^l_{jk}.(48)
\begin{equation}
	\Delta a^l_j \approx \frac{\partial a^l_j}{\partial w^l_{jk}} \Delta w^l_{jk}.
	\tag{48}\label{eq:48}
\end{equation}
The change in activation $\Delta{}a^l_j$ will cause changes in all the activations in the next layer, i.e., the ($l$+1)-th layer. We'll concentrate on the way just a single one of those activations is affected, say $a^{l+1}_q$,
\begin{center}
	\includegraphics[scale=0.5]{./figures/ch2/tikz26.png}
\end{center}
In fact, it'll cause the following change:
\begin{equation}
	\Delta a^{l+1}_q \approx \frac{\partial a^{l+1}_q}{\partial a^l_j} \Delta a^l_j.
	\tag{49}\label{eq:49}
\end{equation}%\Delta{}al+1q≈\partial{}al+1q\partial{}alj\Delta{}alj.(49)
Substituting in the expression from Equation~(\ref{eq:48}), we get:
%\Delta{}al+1q≈\partial{}al+1q\partial{}alj\partial{}alj\partial{}w^l_{jk}\Delta{}w^l_{jk}.(50)
\begin{equation}
	\Delta a^{l+1}_q \approx \frac{\partial a^{l+1}_q}{\partial a^l_j} \frac{\partial a^l_j}{\partial w^l_{jk}} \Delta w^l_{jk}.
	\tag{50}\label{eq:50}
\end{equation}
Of course, the change $\Delta{}a^{l+1}_q$ will, in turn, cause changes in the activations in the next layer. In fact, we can imagine a path all the way through the network from $w^l_{jk}$ to $C$, with each change in activation causing a change in the next activation, and, finally, a change in the cost at the output. If the path goes through activations $a^l_j,a^{l+1}_q,\cdots,a^{L-1}_n,a^L_m$ then the resulting expression is
\begin{equation}
	\Delta C \approx \frac{\partial C}{\partial a^L_m} 
	\frac{\partial a^L_m}{\partial a^{L-1}_n}
	\frac{\partial a^{L-1}_n}{\partial a^{L-2}_p} \ldots
	\frac{\partial a^{l+1}_q}{\partial a^l_j}
	\frac{\partial a^l_j}{\partial w^l_{jk}} \Delta w^l_{jk},
	\tag{51}\label{eq:51}
\end{equation}
%\Delta{}C≈\partial{}C\partial{}aLm\partial{}aLm\partial{}aL-1n\partial{}aL-1n\partial{}aL-2p…\partial{}al+1q\partial{}alj\partial{}alj\partial{}w^l_{jk}\Delta{}w^l_{jk},(51)
that is, we've picked up a $\partial{}a/\partial{}a$ type term for each additional neuron we've passed through, as well as the $\partial{}C/\partial{}a^L_m$ term at the end. This represents the change in $C$ due to changes in the activations along this particular path through the network. Of course, there's many paths by which a change in $w^l_{jk}$ can propagate to affect the cost, and we've been considering just a single path. To compute the total change in $C$ it is plausible that we should sum over all the possible paths between the weight and the final cost, i.e.,
\begin{equation}
	\Delta C \approx \sum_{mnp\ldots q} \frac{\partial C}{\partial a^L_m} 
	\frac{\partial a^L_m}{\partial a^{L-1}_n}
	\frac{\partial a^{L-1}_n}{\partial a^{L-2}_p} \ldots
	\frac{\partial a^{l+1}_q}{\partial a^l_j} 
	\frac{\partial a^l_j}{\partial w^l_{jk}} \Delta w^l_{jk},
	\tag{52}\label{eq:52}
\end{equation}
%\Delta{}C≈∑mnp…q\partial{}C\partial{}aLm\partial{}aLm\partial{}aL-1n\partial{}aL-1n\partial{}aL-2p…\partial{}al+1q\partial{}alj\partial{}alj\partial{}w^l_{jk}\Delta{}w^l_{jk},(52)
where we've summed over all possible choices for the intermediate neurons along the path. Comparing with (\ref{eq:47}) we see that
%\partial{}C\partial{}w^l_{jk}=∑mnp…q\partial{}C\partial{}aLm\partial{}aLm\partial{}aL-1n\partial{}aL-1n\partial{}aL-2p…\partial{}al+1q\partial{}alj\partial{}alj\partial{}w^l_{jk}.(53)
\begin{equation}
	\frac{\partial C}{\partial w^l_{jk}} = \sum_{mnp\ldots q} \frac{\partial C}{\partial a^L_m} 
	\frac{\partial a^L_m}{\partial a^{L-1}_n}
	\frac{\partial a^{L-1}_n}{\partial a^{L-2}_p} \ldots
	\frac{\partial a^{l+1}_q}{\partial a^l_j} 
	\frac{\partial a^l_j}{\partial w^l_{jk}}.
	\tag{53}\label{eq:53}
\end{equation}
Now, Equation~(\ref{eq:53}) looks complicated. However, it has a nice intuitive interpretation. We're computing the rate of change of $C$ with respect to a weight in the network. What the equation tells us is that every edge between two neurons in the network is associated with a rate factor which is just the partial derivative of one neuron's activation with respect to the other neuron's activation. The edge from the first weight to the first neuron has a rate factor $\partial{}a^l_j/\partial{}w^l_{jk}$. The rate factor for a path is just the product of the rate factors along the path. And the total rate of change $\partial{}C/\partial{}w^l_{jk}$ is just the sum of the rate factors of all paths from the initial weight to the final cost. This procedure is illustrated here, for a single path:
\begin{center}
	\includegraphics[scale=0.5]{./figures/ch2/tikz27.png}
\end{center}
What I've been providing up to now is a heuristic argument, a way of thinking about what's going on when you perturb a weight in a network. Let me sketch out a line of thinking you could use to further develop this argument. First, you could derive explicit expressions for all the individual partial derivatives in Equation~(\ref{eq:53}). That's easy to do with a bit of calculus. Having done that, you could then try to figure out how to write all the sums over indices as matrix multiplications. This turns out to be tedious, and requires some persistence, but not extraordinary insight. After doing all this, and then simplifying as much as possible, what you discover is that you end up with exactly the backpropagation algorithm! And so you can think of the backpropagation algorithm as providing a way of computing the sum over the rate factor for all these paths. Or, to put it slightly differently, the backpropagation algorithm is a clever way of keeping track of small perturbations to the weights (and biases) as they propagate through the network, reach the output, and then affect the cost.

\end{document}
